\ssr{АНАЛИТИЧЕСКАЯ ЧАСТЬ}

\section*{Постановка задачи}
Рассматривается ориентированный взвешенный граф $G = (V, E)$, описанный во входном файле в
формате Graphviz DOT. Мощность множества вершин превышает 3000, степени вершин ограничены
значениями от 0 до 15. Необходимо найти маршрут коммивояжёра минимальной длины, проходящий
через каждую вершину ровно один раз и возвращающийся в исходную вершину. В случае
существования нескольких оптимальных маршрутов требуется сохранить их все.

\section*{Последовательный алгоритм}
В качестве базового метода выбран классический муравьиный алгоритм. Каждая особь
последовательно строит допустимый маршрут, выбирая следующую вершину с вероятностью,
пропорциональной сочетанию текущих феромонов и эвристики (обратной величине веса дуги).
После завершения итерации феромоны частично испаряются и пополняются вкладом всех
успешных муравьёв. Алгоритм повторяется фиксированное число итераций; лучшая найденная
комбинация маршрутов сохраняется.

\section*{Параллельный алгоритм}
Параллельная модификация основана на независимом построении маршрутов группами муравьёв,
закреплённых за рабочими потоками. Каждый поток обслуживает собственный набор муравьёв и
использует локальные матрицы вкладов феромонов, исключающие необходимость совместного
доступа при построении маршрутов. После завершения работы всех потоков их вклад
объединяется главной нитью, что позволяет обновить глобальную матрицу феромонов.

\section*{Средства синхронизации}
При параллельном выполнении требуется синхронизация только на этапе обновления глобального
списка оптимальных маршрутов. Для защиты этой структуры данных применяется мьютекс
(std::mutex), обеспечивающий взаимное исключение при добавлении новых маршрутов.
Обновление матрицы феромонов выполняется без синхронизации за счёт предварительного
аккумулирования вкладов в локальных буферах потоков.

\section*{Основные понятия}
\begin{itemize}
        \item \textbf{Поток} --- единица распараллеливания в операционной системе, имеющая собственный стек и регистры, но разделяющая адресное пространство процесса.
        \item \textbf{Мьютекс} --- примитив взаимного исключения, предоставляющий единовременный доступ к разделяемому ресурсу одному потоку.
        \item \textbf{Семафор} --- синхронизационный объект, ограничивающий количество потоков, одновременно обращающихся к ресурсу (в работе не использовался).
\end{itemize}

\clearpage
