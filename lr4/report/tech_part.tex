\ssr{ТЕХНОЛОГИЧЕСКАЯ ЧАСТЬ}

\section*{Средства разработки}
Реализация выполнена на языке C++17 с использованием стандартной библиотеки. Для работы с
потоками применён класс \texttt{std::thread}, для синхронизации --- \texttt{std::mutex}. Ввод
графа осуществляется через стандартные потоки ввода-вывода, для разбора формата DOT
используются регулярные выражения из заголовка \texttt{<regex>}.

\section*{Структура проекта}
Исходные коды расположены в каталоге \texttt{code}. Файл \texttt{graph.cpp} содержит реализацию
класса графа и парсер DOT, \texttt{ant\_colony\_solver.cpp} --- реализацию последовательного и
параллельного алгоритмов, \texttt{main.cpp} --- точку входа утилиты командной строки. В каталоге
\texttt{code/tests} размещён модульный тест, проверяющий корректность парсера и согласованность
результатов параллельной и последовательной версий.

\section*{Сборка и запуск}
Для автоматизации сборки используется \texttt{make}. Цель \texttt{make test} собирает и запускает
модульные тесты, формируя отчёт о прохождении. Генерация отчёта реализована в каталоге
\texttt{report} при помощи \texttt{pdflatex}.

\section*{Особенности реализации}
\begin{itemize}
        \item Потоки создаются вручную для каждой итерации алгоритма, что удовлетворяет требованию отказа от стандартных пулов.
        \item Кандидатные маршруты нормализуются при помощи канонического представления, позволяющего хранить только уникальные циклы оптимальной длины.
        \item Для повышения детерминированности применяются индивидуальные сдвиги начальных зёрен генераторов случайных чисел для каждого потока и итерации.
\end{itemize}

\clearpage
