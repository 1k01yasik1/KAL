\ssr{ВВЕДЕНИЕ}

\textbf{Цель работы:} сравнение алгоритмов умножения матриц стандартного и Винограда с оценкой трудоёмкости и времени выполнения.

\textbf{Задачи лабораторной работы.}
\begin{enumerate}
	\item Описать теоретические основы стандартного алгоритма умножения матриц и алгоритма Винограда.
	\item Разработать модель вычислений.
	\item На основе модели вычислений оценить трудоёмкость трёх алгоритмов умножения матриц, учитывая чётный (лучший) и нечётный (худший) случаи.
	\item Реализовать программное обеспечение с двумя режимами работы:
	\begin{itemize}
		\item одиночное умножение матриц, введённых пользователем;
		\item массированные замеры времени выполнения алгоритмов при варьируемом линейном размере квадратных матриц.
	\end{itemize}
	\item Выполнить замеры процессорного времени выполнения реализованных алгоритмов умножения матриц и построить графики зависимости времени выполнения от линейного размера матриц.
	\item Провести сравнительный анализ стандартного алгоритма, алгоритма Винограда, оптимизированную версию алгоритма Винограда, на основе полученных результатов измерений.
\end{enumerate}


\clearpage
