\ssr{ЗАКЛЮЧЕНИЕ}

В ходе работы были рассмотрены три метода умножения матриц: стандартный алгоритм, алгоритм Винограда и его оптимизированная модификация. 
Для каждого метода выполнен анализ трудоёмкости на основе модели вычислений, а также реализованы программные версии для проверки их эффективности на практике.

Теоретический анализ показал, что все рассмотренные алгоритмы имеют одинаковую асимптотическую сложность $O(N^3)$, однако различаются по количеству выполняемых базовых операций. Стандартный алгоритм является исходным вариантом для сравнения, алгоритм Винограда уменьшает количество умножений за счёт предварительных вычислений, а оптимизированная версия дополнительно снижает трудоёмкость благодаря буферизации и повторному использованию промежуточных данных.

Проведённые вычислительные исследования подтвердили результаты теоретического анализа. На тестовых примерах с различными размерами матриц алгоритм Винограда показал преимущество по скорости перед стандартным методом, а оптимизированный вариант продемонстрировал наилучшие показатели времени выполнения. Также отмечено влияние чётности размеров матриц: при нечётных значениях время работы увеличивается, что соответствует теоретическим оценкам.

Цель работы — исследование и сравнение эффективности алгоритмов умножения матриц — достигнута. 
В ходе выполнения работы были решены следующие задачи:

\begin{itemize}
	\item были рассмотрены теоретические основы стандартного алгоритма и алгоритма Винограда;
	\item была разработана модель вычислений для оценки трудоёмкости;
	\item на основе модели вычислений был выполнен анализ трудоёмкости алгоритмов в лучшем и худшем случаях;
	\item было реализовано программное обеспечение, включающее два режима работы:
	\begin{itemize}
		\item режим ручного ввода матриц и вывода результата умножения;
		\item режим автоматического измерения времени выполнения алгоритмов при изменении размера матриц;
	\end{itemize}
	\item были произведены измерения процессорного времени выполнения алгоритмов и построены графики зависимости времени от размера матриц;
	\item было выполнено сравнение эффективности стандартного алгоритма, алгоритма Винограда и его оптимизированной версии на основе полученных данных.
\end{itemize}

\clearpage
