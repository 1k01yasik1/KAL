\chapter{Аналитическая часть}

\section{Алгоритмы умножения матриц}

\subsection{Стандартный алгоритм умножения матриц}

Стандартный алгоритм реализует прямое определение операции матричного произведения, соответствующее основам линейной алгебры.~\cite{lit1} Пусть заданы матрицы $A$ размером $n \times m$ и $B$ размером $m \times p$. Результирующая матрица $C$ имеет размер $n \times p$, а её элементы определяются как скалярные произведения соответствующих строк матрицы $A$ и столбцов матрицы $B$. Вычисления выполняются по формуле~\eqref{eq:standard}.
\begin{equation} \label{eq:standard}
	C[i][j] = \sum_{k=1}^{m} A[i][k] \cdot B[k][j], \quad i = 1..n, \; j = 1..p
\end{equation}

Реализация алгоритма предполагает использование трёх вложенных циклов: внешний цикл перебирает строки первой матрицы, средний — столбцы второй, а внутренний выполняет суммирование произведений соответствующих элементов.

\subsection{Алгоритм Винограда}

Метод Винограда представляет собой усовершенствование стандартного подхода, позволяющее уменьшить число умножений за счёт частичного предварительного вычисления.~\cite{lit2} Для каждой строки матрицы $A$ и каждого столбца матрицы $B$ заранее формируются вспомогательные значения, представленные в формулах \eqref{eq:vinograd_row} и \eqref{eq:vinograd_col}.
\begin{equation} \label{eq:vinograd_row}
	row[i] = \sum_{k=1}^{\lfloor m/2 \rfloor} A[i][2k-1] \cdot A[i][2k]
\end{equation}
\begin{equation} \label{eq:vinograd_col}
	col[j] = \sum_{k=1}^{\lfloor m/2 \rfloor} B[2k-1][j] \cdot B[2k][j]
\end{equation}

После этого каждый элемент результирующей матрицы вычисляется по формуле \eqref{eq:vinograd}.
\begin{equation} \label{eq:vinograd}
	C[i][j] = -row[i] - col[j] + \sum_{k=1}^{\lfloor m/2 \rfloor} (A[i][2k-1] + B[2k][j]) \cdot (A[i][2k] + B[2k-1][j])
\end{equation}

Если $m$ нечётно, то добавляется дополнительное слагаемое \eqref{eq:vinograd_extra}:
\begin{equation} \label{eq:vinograd_extra}
	C[i][j] = C[i][j] + A[i][m]\cdot B[m][j]
\end{equation}

За счёт введения вспомогательных сумм количество умножений в формуле \eqref{eq:vinograd} уменьшается по сравнению со стандартным алгоритмом.

\section*{Вывод}

В данном разделе были рассмотрены два основных алгоритма умножения матриц — стандартный и алгоритм Винограда. Стандартный метод реализует прямое определение операции матричного умножения и требует тройного цикла вычислений, что приводит к высокой трудоёмкости при увеличении размерности матриц.

Алгоритм Винограда представляет собой оптимизацию стандартного подхода за счёт предварительного вычисления частичных произведений строк и столбцов. Это позволяет сократить количество умножений в основной части алгоритма, что особенно заметно при больших размерах матриц. Таким образом, метод Винограда достигает меньшей константы в выражении трудоёмкости при сохранении той же асимптотической сложности $O(N^3)$.






\clearpage
